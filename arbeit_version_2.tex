%! Author = Lukas Bilstein
%! Date = 01.09.2021

% Preamble
\documentclass[12pt]{article}

% Packages
\usepackage{amsmath}
\usepackage{setspace}
\usepackage{parskip}
\usepackage{amsfonts}
\usepackage[ngerman]{babel}

\usepackage[a4paper, left=3.5cm, right=2.5cm, top=2.5cm, bottom=2cm]{geometry}
\usepackage{mathtools}


% Setup spacing
\onehalfspacing


% Document
\begin{document}

    \thispagestyle{empty}

    {
        \centering

        \begin{singlespace}

        \textbf{Hans-Sachs-Gymnasium Nürnberg}

        \textbf{Oberstufenjahrgang 2020/2022}

        \end{singlespace}

        \bigskip

        Seminarfach Mathematik

        \bigskip

        Seminararbeit

        \bigskip
        \bigskip

        \textbf{Reelle Zahlen und ihre Arithmetik in der Darstellung der dedekind'schen Schnitte}

        \bigskip
        \bigskip

    }

    \hspace*{30mm}Verfasser: \hspace*{22mm} Lukas Bilstein

    \bigskip

    \hspace*{30mm}Kursleiter: \hspace*{20mm} OStR Dr. Dennis Simon

    \bigskip

    \hspace*{30mm}Bewertung der Arbeit: \hspace*{10mm} \line(1, 0){35mm}

    \bigskip

    \hspace*{30mm}Bewertung der Präsentation: \line(1, 0){35mm}

    \bigskip
    \bigskip
    \bigskip
    \bigskip
    \bigskip

    \hspace*{30mm}Unterschrift Kursleiter: \hspace*{8mm} \line(1, 0){45mm}

    \newpage

    \thispagestyle{empty}

    \tableofcontents

    \newpage

    \setcounter{page}{3}

    \section{Unvollständigkeit der rationale Zahlen}

    Nach der Definition der Multiplikation in den ganzen Zahlen $\mathbb{Z}$, stellten wir fest, dass
    die Umkehrung der Multiplikation, die Division, nicht für alle Paare ganzer Zahlen definiert ist.
    So ist $\frac{1}{2}$ keine ganze Zahl.

    Daraufhin haben wir die rationalen Zahlen eingeführt, welche ebendiese Lücke gefüllt haben.

    Doch auch die rationalen Zahlen sind auch nicht vollständig.
    Die Gleichung $a^2 = a * a = 2$ hat keine Lösung in den rationalen Zahlen.
    
    Beweis:
    Gibt es eine rationale Zahl q mit $q = \frac{m}{n}$, sodass $q = \sqrt{2}$, so muss gelten:
    \begin{equation}
        q = \sqrt{2} \Leftrightarrow \frac{m}{n} = \sqrt{2} \Leftrightarrow \frac{m^2}{n^2} = 2 \Leftrightarrow m^2 = 2n^2
    \end{equation}

    Also muss entweder $m^2$ oder $n^2$ ungerade viele 2-er in der
    Primzahlenzerlegung enthalten, was aber unmöglich ist, da Quadratzahlen Primfaktoren immer gerade oft enthalten.
    
    Damit kann es keine rationale Zahl q geben, die die obige Gleichung erfüllt.

    Weiterhin gilt dies für alle rationalen Zahlen, welche sich nicht als Quotient zweier Quadratzahlen darstellen
    lassen.

    \section{Näherung von $\sqrt{2}$}

    Wir können aber klar definieren, welche rationalen Zahlen kleiner bzw. größer als $\sqrt{2}$ sind:

    \[\{n \in \mathbb{N} \mspace{4mu} | \mspace{4mu} n < 0 \lor n^2 < 2\}\]

    \[\{n \in \mathbb{N} \mspace{4mu} | \mspace{4mu} n > 0 \land n^2 > 2\}\]

    An dieser Stelle müssen wir die Bedingung $n > 0$ bzw. $n < 0$ einfügen, da $n ^ 2 = 2$ zwei Lösungen besitzt,
    an dieser Stelle aber nur die Positive gesucht ist.

    Beide Mengen sind klar definiert, enthalten aber weder gemeinsame Elemente noch $\sqrt{2}$.
    (Da diese keine rationale Zahl ist)

    Wir wollen im folgendem die `obere` Menge als Dedekind'scher Schnitt bezeichnen.

    \section{Formale Defintion Dedekind'scher Schnitte}

    1. Ein Dedekind'scher Schnitt r ist eine nicht-leere Teilmenge der rationalen Zahlen $\mathbb{Z}$,
    also ein Element der Potenzmenge $\mathcal{P}(\mathbb{Z})$, welche nach unten beschränkt ist,
    (Vgl. Begriffsdefinition im nachfolgendem Teil) und nicht gleich den rationalen Zahlen selbst ist.

    2. Es muss gelten: $\forall a \in r: \forall b \in \mathbb{Z}: b \ge a \rightarrow b \in r$, d.h.
    für eine rationale Zahl in einem dedekind'schen Schnitt müssen alle größeren rationalen Zahlen auch enthalten sein,
    sodass der dedekind'sche Schnitt keine `Lücken` hatt.

    3. Für einen dedekind'schen Schnitt r gibt es keine rationale Zahl q, für die $q \in r$ und $\forall x \in r: q \le x$,
    also r besitzt keine kleinste rationale Zahl.

    Wir definieren nun die Menge $\mathbb{R}$ als die Menge aller dedekind'schen Schnitten, also der Menge aller
    Elemente der Potenzmenge der rationalen Zahlen, für die obige drei Eingenschaften gelten.

    Anmerkung:
    Je nach Literatur wird ein dedekidsch'scher Schnitt auch als eine nach oben Beschränkte Teilmenge beschrieben,
    welche kein größtes Element besitzt.

    \subsection{Übergang von den rationalen Zahlen zu den reellen Zahlen}

    Wir müssen eine Funktion definieren, die jeder rationalen Zahlen q einen dedekind'schen Schnitt r zuordnet.

    Eine solche Funktion ist:

    \[r = \{x > q \mspace{4mu} | \mspace{4mu} x \in \mathbb{Q}\}\]

    Im folgenden werden wir sehen, dass diese Definition die Eigenschaften der rationalen Zahlen erhält, wenn
    wir deren dedekind'sche Schnitt betrachten.

    Das Infimum eines solchen Schnittes ist die rationale Zahl selbst, die diesen Schnitt beschreibt.
    Wir werden später feststellen, das nicht alle dedekind'schen Schnitte ein rationales Infinum besitzen.

    \section{Begriffsdefinitionen}

    Zu einer Teilmenge m einer Menge M mit einer Ordnung, wie sie die Menge der rationalen Zahlen darstellt,
    kann es obere und untere Schranken geben.

    Eine obere Schranke s ist ein Element aus M, für das alle Elemente aus m kleiner als s sind.

    Eine untere Schranke analog für das alle Element aus m größer als s sind.

    Besitzt M eine Schranke, spricht mann dass M nach der jeweiligen Seite beschränkt ist.

    Weiterhin ist das Supremum die kleinste obere Schranke, und das Infimum die größte untere Schranke einer Menge m.

    \subsection{Eigenschaften von Mengendefinitionsverschachtelungen}

    Sei A definiert durch $A = \{f(a) \mspace{4mu} | \mspace{4mu} a \in \{g(x) \mspace{4mu} | \mspace{4mu}  x \in I\}\}$.

    Wir dürfen A umschreiben als:
    $A = \{f(g(x)) \mspace{4mu} | \mspace{4mu} x \in I\}$

    Ebenfalls gilt die Umkehrung und die Erweiterung auf mehr-variabligen Funktionen.

    \section{Ordnung in den reellen Zahlen}

    Seien a und b zwei dedekind'sche Schnitte. Wir definieren

    \[a > b := \exists x \in b: b \not \in a\]

    Und weiterhin $a < b := b > a$, $a \ge b := \lnot (a < b)$ und $a \le b := \lnot (a > b)$.

    \section{Der Körper der reellen Zahlen}

    Ein Körper K ist ein System aus einer Menge M, und zwei Operationen `+' und `*', die je zwei Elementen aus der Menge
    M einem der gleichen zuordnet.

    Für die Operationen `+' und `*' muss jeweils gelten (wobei $\emptyset$ ein eindeutiges neutrales Element für
    die Operation darstellt):

    - $\forall x, y, t \in M: x \bigoplus (y \bigoplus z) = (x \bigoplus y) \bigoplus z$  (Assoziativ)

    - $\forall x, y \in M: x \bigoplus y = y \bigoplus x$  (Kommutativ)

    - $\exists x \in M: \forall y \in M: x \bigoplus y = y$  (Existenz eines neutralen Elements)

    - $\forall x \in M: \exists y \in M: x \bigoplus y = \emptyset$  (Existenz einer Inversen)

    Und Operationsübergreifend:

    - $\forall x, y, z \in M: (x + y) * z = (x * z) + (y * z)$ (Distributiv)

    % links/rechts-distributiv?

    All diese sollen im folgenden für unser konstruiertes $\mathbb{R}$ bewiesen werden.

    \subsection{Addition dedekind'scher Schnitte}

    Wir definieren zunächst die Addition folgendermaßen:

    $a + b \coloneqq \{x + y \mspace{4mu} | \mspace{4mu} x \in a \land y \in b\}$

    Wir müssen zunächst beweisen, dass $a + b$ tatsächlich ein dedekind'scher Schnitt ist.

    $a + b$ ist nicht leer, da $a + b$ alle Kombinationen aus a und b enthält, und a und b nicht leer sind.

    $a + b$ enthält nicht alle rationalen Zahlen, da die Zahl $infimum(a) + infimum(b) - 1$ nicht in $a + b$
    enthalten sein kann, da dafür ein Element kleiner als das Infimum in a oder b enthalten sein müsste, was
    aber nach der Definition unmöglich ist.

    Es ist ebenfalls klar, das $a + b$ keine Lücken hat.
    Sei m eine Zahl aus dem Schnitt $a + b$ und n eine Zahl die größer als m ist, aber nicht in $a + b$ enthalten ist.
    Sei $x \in a$ und $y \in b$, sodass $x + y = m$. Da a neben x auch die Zahl $x + (n - m)$ enthalten muss,
    muss auch $x + (n - m) + y$ in $a + b$ enthalten sein. Ersetzen von $x + y$ durch m ergibt $n - m + m = n$,
    was bedeutet, dass n auch in $a + b$ enthalten sein muss, was aber im widerspruch zu unser Annahme steht.

    Sei $x + y$ das kleinstes Element in $a + b$. Sei weiterhin x' ein Element aus a.
    Sei $y' = x + y - x'$. Diese Zahl kann nicht $\in b$ sein, da sonst $x' + y' > x + y \Leftrightarrow y' > x + y - y'$.
    Das steht im Widerspruch zur Annahme, dass $x + y \in a + b$, da per Definition der Addition nur solche Elemente
    in $a + b$ enthalten sind, für die $x \in a$ und $y \in b$.

    \subsubsection{Kommutativität und neutrales Element}

    Die Kommutativität folgt aus der Kommutativität der rationalen Zahlen:

    \[a + b = \{x + y \mspace{4mu} | \mspace{4mu} x \in a \land y \in b\} =\]
    \[= \{y + x \mspace{4mu} | \mspace{4mu} x \in a \land y \in b\} =\]
    \[= \{y + x \mspace{4mu} | \mspace{4mu} y \in b \land x \in a\} =\]
    \[= b + a\]

    Das neutrale Element ist die Menge $\{a > 0 \mspace{4mu} | \mspace{4mu} a \in \mathbb{Q}\}$.
    Bei Addition kommen nur Elemente größer-gleich der ursprünglichen Elemente heraus, und damit bleibt
    der dedekidsch'sche Schnitt identisch.

    \subsubsection{Assoziativität}

    Wir können $A + (B + C)$ schreiben als:
    \[\{a+d \mspace{4mu} | \mspace{4mu} a \in A \land d \in \{b + c \mspace{4mu} | \mspace{4mu} b \in B \land c \in C\}\}\]

    Was wir umschrieben dürfen als:
    \[\{a+b+c \mspace{4mu} | \mspace{4mu} a \in A \land b \in B \land c \in C\}\]

    Und weiterhin:

    \[\{d+c \mspace{4mu} | \mspace{4mu} d \in \{a + b \mspace{4mu} | \mspace{4mu} a \in A \land b \in B\} \land c \in C\}\]

    Was identisch zur gefordertem Eigenschaft $(A + B) + C$ ist.

    % Dies könnten wir hier noch ausführen, wenn wir noch was brauchen

    \subsubsection{Inverse Element und Subtraktion}

    Wir schreiben das Inversum von A als $-A$ und definieren es als:

    \[A^- \coloneqq \{b \in \mathbb{Q} \mspace{4mu} | \mspace{4mu} a + b > 0 \land a \in A\}\]

    Die Menge ist nicht-leer, da $a + b > 0$ in den rationalen Zahlen Lösungen besitzt, solange $|a| \not = \infty$ und
    $|b| \not = \infty$.

    Aus gleichem Grund ist die Menge nicht gleich den rationalen Zahlen, da es für obige Gleichung rationale Zahlen gibt,
    die diese nicht erfüllen.

    Doch das Inversum kann ein kleinstes Element haben. In diesem Fall müssen wir dieses Element ausschließen,
    und wir definieren also:

    $-A := \begin{cases}A^- \mspace{6mu} wenn \mspace{6mu} A^- \mspace{6mu} kein \mspace{6mu} kleinstes \mspace{6mu} Element \mspace{6mu} hat \\A^- \symbol{'134} {x_0} \mspace{6mu} sonst\end{cases}$

    Die Subtraktion ist wie üblich Definiert mit:
    $A - B := A + (-B)$

    Weiterhin ist damit der Betrag eines dedekindsch'schen Schnittes definiert als:
    \[|A| := \begin{cases}A \mspace{10mu} wenn \mspace{6mu} A \ge 0 \\ -A \mspace{10mu} sonst\end{cases}\]

    \subsection{Multiplikation dedekidsch'scher Schnitte}

    Wir definieren $A \times B$ für zwei positive dedekidsch'sche Schnitte A und B folgendermaßen:

    \[A \times B \coloneqq \{a * b \mspace{4mu} | \mspace{4mu} a \in A \land b \in B\]

    Diese Menge ist wieder nicht-leer, da sie wieder die Kombinationen zweier nicht-leerer Mengen enthält.

    Sie ist nicht identisch zu den rationalen Zahlen, da sie nur positive Elemente enthalten kann.

    Sie hat kein kleinstes Element.

    Wir können nun diese Multiplikation auch auf negative Schnitte erweitern, dafür verwenden wir folgende
    Fallunterscheidung:
    \[
        A * B \coloneqq
        \begin{cases}
            A \times B \mspace{4mu} wenn \mspace{4mu} A \ge 0 \mspace{4mu} und \mspace{4mu} B \ge 0 \\
            -A \times -B \mspace{4mu} wenn \mspace{4mu} A < 0 \mspace{4mu} und \mspace{4mu} B < 0 \\
            -(-A \times B) \mspace{4mu} wenn \mspace{4mu} A < 0 \mspace{4mu} und \mspace{4mu} B \ge 0  \\
            -(A \times -B) \mspace{4mu} wenn \mspace{4mu} A \ge 0 \mspace{4mu} und \mspace{4mu} B < 0
        \end{cases}
    \]

    Die Kommutativität für positive Schnitte ist wie bei der Addition klar ersichtlich.
    Ein Vorzeichen ändert daran nichts, da dieses in der eigentlichen Berechung nicht verwendet wird.

    \subsubsection{Assoziativität}

    Seien a, b und c drei dedekind'sche Schnitte, und d = (a * b) * c.
    Der einfachheit halber seien a, b und c zunächst positiv.

    Wir können d schreiben als:

    \[d = \{m * n \mspace{4mu} | \mspace{4mu} m \in \{x * y \mspace{4mu} | \mspace{4mu} x \in a \land y \in b\} \land n \in c\}\]

    Und durch Vereinfachung:

    \[d = \{(x * y) * n \mspace{4mu} | \mspace{4mu} (x \in a \land y \in b) \land n \in c\}\]

    Und weiterhin:

    \[d = \{x * y * n \mspace{4mu} | \mspace{4mu} x \in a \land y \in b \land n \in c\}\]

    Was sich umformen lässt zu:

    \[d = \{x * m \mspace{4mu} | \mspace{4mu} x \in a \land m \in \{y * n \mspace{4mu} | \mspace{4mu} y \in b \land n \in c\}\}\]

    Auf dem selben weg kommen wir zur geforderten Gleichung $d = a * (b * c)$.

    \subsubsection{Neutrales Element}

    Das neutrale Element der Multiplikation ist die `1`, also $\{q > 1 \mspace{4mu} | \mspace{4mu} q \in \mathbb{Q}\}$

    Dabei ist klar, dass für einen positiven Schnitt a bei Multiplikation mit 1 für jedes Element aus a nur Elemente
    größer-gleich demselben entstehen, und damit keine kleineren Elemente dazukommen, sodass der Schnitt gleich
    bliebt.

    \subsubsection{Inverse Element und die Division}

    Wir definieren:
    \[a* := \{y \mspace{4mu} | \mspace{4mu} x * y > 1 \land x \in a \land y \in \mathbb{Q}\}\]

    a* ist nicht-leer und enthält nicht alle Elemente der rationalen Zahlen.

    Aber wir stellen fest, dass diese Menge ein kleinstes Element haben kann.

    Wir definieren also $a^{-1}$ als:

    \[
        a^{-1} \coloneqq
        \begin{cases}
            \{y \mspace{4mu} | \mspace{4mu} x * y > 1 \land x \in a \land y \in \mathbb{Q}\} \mspace{4mu} wenn
            \mspace{4mu} diese \mspace{4mu} kein \mspace{4mu} kleinstes \mspace{4mu} Element \mspace{4mu} hat \\
            \{y \mspace{4mu} | \mspace{4mu} x * y > 1 \land x \in a \land y \in \mathbb{Q}\} \symbol{'134} \{kleinstes Element\} \mspace{4mu} sonst
        \end{cases}
    \]

    Und weiterhin definieren wir die Division mithilfe:

    $a \div b := a * b^{-1}$

    \subsection{Distributivität}

    Wir betrachten den Term $(A + B) * C$.
    Für den Fall, dass $A, B, C > 0$ sind, gilt:

    \[
        (A + B) * C = \{x * c \mspace{4mu} | \mspace{4mu} x \in \{a + b \mspace{4mu} | \mspace{4mu} a \in A \land b \in B\} \land c \in C\} =
    \]
    \[
        = \{(a + b) * c \mspace{4mu} | \mspace{4mu} a \in A \land b \in B \land c \in C\} =
    \]
    \[
        = \{a * c + b * c \mspace{4mu} | \mspace{4mu} a \in A \land b \in B \land c \in C\} =
    \]
    \[
        = \{x + y \mspace{4mu} | \mspace{4mu} x \in \{a * c \mspace{4mu} | \mspace{4mu} a \in A \land c \in C\} \land
        \{b * c \mspace{4mu} | \mspace{4mu} b \in B \land c \in C\}\} =
    \]
    \[
        = (A * C) + (B * C)
    \]

    Wenn C negativ ist, substituieren wir $C \rightarrow -C$ und erhalten:

    \[
        = (A + B) * (-1) * C = (-1) * ((A + B) * C) = (-1) * (AB + AC) =
    \]

    Wir setzten nun $A' = AC$ und $B' = BC$ und erhalten mit Substitution der Inversionsformel:

    \[
        = \{b \in \mathbb{Q} \mspace{4mu} | \mspace{4mu} x + b > 0 \land x \in (A' + B')\} =
    \]
    \[
        = \{b \in \mathbb(Q) \mspace{4mu} | \mspace{4mu} a' + b' + b > 0 \land a' \in A' \land b' \in B'\} =
    \]
    \[
        = \{b_1 + b_2 \in \mathbb{Q} \mspace{4mu} | \mspace{4mu} a' + b_1 > 0 \land b' + a_2 > 0 \land a' \in A' \land b' \in B'\} =
    \]

    Bei Aufspaltung durch die Rechenregeln dedekind'scher Schnitte:

    \[
        = \{b_1 \in \mathbb{Q} \mspace{4mu} | \mspace{4mu} a' + b_1 > 0 \land a' \in A'\} + \{b_2 \in \mathbb{Q} \mspace{4mu} | \mspace{4mu} b' + b_2 > 0 \land b' \in B'\} =
    \]

    Und Verwendung der Inversionsformel:

    \[
        = (-A') + (-B') = A*(-C) + B * (-C)
    \]

    Und resubstituiert $C \rightarrow -C$ die geforderte Formel $AC + BC$

    Wenn A oder B $< 0$ ist, nehmen wur an, es sei A, substituieren wir $A \rightarrow -A$ und erhalten $(-A)C+BC$,
    und können wieder re-substituieren zu $AC + BC$. Gleichermaßen wenn A und B negativ sind.

    Damit ist unser Körper wirklich ein Körper.

    \section{Abgrenzung zu den rationalen Zahlen}

    Nachdem wir anfangs herausgefunden hatten, dass Zahlen wie $\sqrt{2}$ keine rationalen Zahlen sind, stellt sich
    die Frage, ob wir diese Eigenschaft nicht mathematisch festhalten können.

    Dafür betrachten wir unser Beispiel, den dedekind'schen Schnitt zu $\sqrt{2}$.
    Diese Menge besitzt kein Infimum, da die Zahl $\sqrt{2}$, welche nicht Teil des dedekind'schen Schnittes ist,
    keine rationale Zahl ist, sich aber beliebig annähern lässt, und somit zu jeder rationalen Zahl kleiner als
    $\sqrt{2}$ eine existiert, die näher an $\sqrt{2}$ ist.

    Wir definieren darauf den Begriff der Vollständigkeit:

    Eine geordnete Menge M ist genau dann Vollständig, wenn alle Teilmengen von M, die nach unten beschränkt sind,
    ein Infimum besitzen.

    Für die Menge $\mathbb{Q}$ ist der Schnitt $\sqrt{2}$ ein Gegenbeispiel, der nach Definition
    der dedekind'schen Schnitte eine Teilmenge von $\mathbb{Q}$ ist.
    Also ist $\mathbb{Q}$ nicht vollständig.

    \subsection{Die Vollständigkeit der reellen Zahlen}

    Sei M eine nicht-leere nach unten beschränkte Teilmenge von $\mathbb{R}$, und $C \in \mathbb{R}$ eine untere Schranke von M.

    Sei
    \[I = \bigcup \limits_{a \in M} a \subseteq M\]

    Da alle Elemente aus M dicht sind, und nach oben bis $\infty$ gehen, muss I ebenfalls dicht sein.
    I besitzt weiterhin kein kleinstes Element, da die kleinsten Elemente aus dem kleinsten Schnitt aus M kommen.
    I ist nach unten beschränkt, da alle Elemente aus M nach unten beschränkt sind.

    Also muss I wieder ein dedekind'scher Schnitt sein.

    Es gilt $I \le a$ für alle $a \in M$, da $\forall m \in M: m \subseteq I$. (Eigenschaften der Vereinigungsmenge)

    % Hier noch etwas genauer...
    Durch die Definition von I folgt, dass I eine Teilmenge von C ist, und damit alle C kleiner als I.
    Damit ist I die größte untere Schranke der Menge M.

    Damit ist $\mathbb{R}$ vollständig, da mit obigen Konstruktion allen reellen Zahlen ein Infimum, I, zugewiesen werden
    können.

    \section{Der Archimedische Körper der reellen Zahlen}

    Ein geordneter Zahlenkörper M ist dann archimedisch, wenn für all seine Elemente m eine natürliche Zahl n existiert,
    die größer als m.
    Dafür benötigt M eine Funktion $f: \mathbb{N} \rightarrow \mathbb{R}$,
    welche einer natürlichen Zahl ein Element aus M zuordnet, um den Vergleich durchzuführen.

    Für die reellen Zahlen ist eine solche Funktion die folgende:

    \[f: n \rightarrow \{q \mspace{4mu} | \mspace{4mu} q \in \mathbb{Q} \land q > n\}\]

    Sei nun N die Menge aller natürlicher Zahlen kleiner als m sind, also
    \[N = \{n \in \mathbb{N} \mspace{4mu} | \mspace{4mu} f(n) \le m\}\]

    Sei s nun das Supremum von N.
    Es muss gelten: $f(s) \le m$.
    Aber auch $f(s+1) \ge m$ und ferner $f(s+2) > m$, da sonst $s+1$ auch Teil von N sein müsste.

    Somit gibt es zu jeder reellen Zahlen m mindestens eine natürliche Zahl n (hier z.B. s + 1), die größer als m ist,
    und  damit ist $\mathbb{R}$ archimedisch.

    \section{Ausblick}

    In der Einleitung haben wir $\sqrt{2}$ betrachtet, und gemerkt, dass sie keine rationale Zahl ist.
    Dies lässt sich einfach auf alle weiteren positiven rationale Zahlen erweitern, wobei nur die Quadrate von rationalen Zahlen
    rationale Wurzeln besitzen.

    Doch der dedekind'sche Schnitt für -1, also $\{a \mspace{4mu} | \mspace{4mu} a^2 > -1\}$, ist gleich $\mathbb{Q}$,
    da dass Quadrat einer rationalen Zahl nie negativ sein kann.
    Wir können an dieser Stelle auch nicht einfach wieder `dedekind'sche Schnitte` über den reellen Zahlen bilden,
    da die reellen Zahlen ebenfalls die Eigenschaft besitzen, dass ihr Quadrat immer postitiv ist.

    (An dieser Stelle sei angemekt, dass eine Konstruktion von dedekind'schen Schnitten über $\mathbb{R}$ gar
    keine neuen Zahlen konstruiert)

    Wir wollen also eigentlich neue Zahlen konstruieren, für die Wurzeln auf beliebige Zahlen, nicht nur postitive,
    definiert sind.

    Desweiteren stellt sich nun das Problem, das durch den Unterschied zwischen $\mathbb{Q}$ und $\mathbb{R}$
    durch die Vollständigkeit es möglich wird, das $\mathbb{R}$ `größer` als $\mathbb{Q}$ ist.

    Dazu kennen wir das Konzept des `Betrages` einer Menge, aber wie vergleichen wir unendlich große Mengen?

    \newpage

    \section{Literaturverzeichnis}

    - Uwe Storch, Hartmut Wiebe: Grundkonzepte der Mathematik: Mengentheoretische, algebraische, topologische Grundlagen
    sowie reelle und komplexe Zahlen, Springer Spektrum Verlag

    - Prof. Dr. Siegfried Echterhoff: Ergänzung zur Vorlesung Analysis I WS08/0, Konstruktion von $\mathbb{R}$,
    https://ivv5hpp.uni-muenster.de/u/echters/Analysis/AnalysisI/Skript/
    Konstruktion\_reelle\_Zahlen.pdf, zuletzt abgerufen am 23.10.2021

    - http://www.math.columbia.edu/~harris/2000/2016Dedcuts.pdf, zuletzt abgerufen am 26.10.2021

    \newpage

    \fbox{\parbox{\linewidth}{

    \textit{Ich erkläre hiermit, dass ich die Seminararbeit ohne fremde Hilfe angefertigt und nur die im
    Literaturverzeichnis angeführten Quellen und Hilfsmittel benützt habe.}

    \bigskip

    ........................................, den ......................................
    \\[-0.2cm]\hspace*{22mm} Ort \hspace*{55mm} Datum

    \bigskip
    \bigskip

    ................................................................................................................
    \\[-0.2cm]Unterschrift des Verfassers
    }}


\end{document}
