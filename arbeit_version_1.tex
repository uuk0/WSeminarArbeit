%! Author = Lukas Bilstein
%! Date = 01.09.2021

% Preamble
\documentclass[12pt]{article}

% Packages
\usepackage{amsmath}
\usepackage{setspace}
\usepackage{parskip}
\usepackage{amsfonts}
\usepackage[ngerman]{babel}

\usepackage[a4paper, left=3.5cm, right=2.5cm, top=2.5cm, bottom=2cm]{geometry}


% Setup spacing
\onehalfspacing


% Document
\begin{document}

    \thispagestyle{empty}

    {
        \centering

        \begin{singlespace}

        \textbf{Hans-Sachs-Gymnasium Nürnberg}

        \textbf{Oberstufenjahrgang 2020/2022}

        \end{singlespace}

        \bigskip

        Seminarfach Mathematik

        \bigskip

        Seminararbeit

        \bigskip
        \bigskip

        \textbf{Reelle Zahlen und ihre Arithmetik in der Darstellung der dedekind'schen Schnitten}

        \bigskip
        \bigskip

    }

    \hspace*{30mm}Verfasser: \hspace*{22mm} Lukas Bilstein

    \bigskip

    \hspace*{30mm}Kursleiter: \hspace*{20mm} OStR Dr. Dennis Simon

    \bigskip

    \hspace*{30mm}Bewertung der Arbeit: \hspace*{10mm} \line(1, 0){35mm}

    \bigskip

    \hspace*{30mm}Bewertung der Präsentation: \line(1, 0){35mm}

    \bigskip
    \bigskip
    \bigskip
    \bigskip
    \bigskip

    \hspace*{30mm}Unterschrift Kursleiter: \hspace*{8mm} \line(1, 0){45mm}

    \newpage

    \thispagestyle{empty}

    \tableofcontents

    \newpage

    \setcounter{page}{3}

    \section{Unvollständigkeit der rationale Zahlen}

    Die rationalen Zahlen (\begin{math}\mathbb{Q}\end{math}) wurden zur Vervollständigung der ganzen Zahlen
    (\begin{math}\mathbb{Z}\end{math}) definiert, in der die
    Umkehrfunktion der Multiplikation, die Division, nicht vollständig definiert ist.

    Doch auch die rationalen Zahlen sind auch nicht vollständig.
    Mit Einführung der Potenz und deren Umkehrung, der Wurzel, stellt sich das Problem, das z.B.
    \begin{math}\sqrt{2}\end{math} keine rationale Zahl ist.
    Beweis:
    Gibt es eine rationale Zahl q mit \begin{math}q=m/n\end{math}, sodass \begin{math}q=\sqrt{2}\end{math}, so muss gelten:
    \begin{equation}
        q = \sqrt{2} \Leftrightarrow m/n = \sqrt{2} \Leftrightarrow m^2/n^2=2 \Leftrightarrow m^2 = 2n^2
    \end{equation}
    Also muss entweder \begin{math}m^2\end{math} oder \begin{math}n^2\end{math} ungerade viele 2-er in der
    Primzahlenzerlegung enthalten, was aber unmöglich ist, da Quadratzahlen Primfaktoren immer gerade oft enthalten.
    Damit kann es keine rationale Zahl q geben, die die genannten Eigenschaften erfüllt, und damit kann auch
    \begin{math}\sqrt{2}\end{math} keine rationale Zahl sein.

    % Hier noch ein Beweis dessen?

    \section{Näherung von \begin{math}\sqrt{2}\end{math}}

    Wir können aber klar definieren, welche Zahlen kleiner und größer als \begin{math}\sqrt{2}\end{math} sind:
    \begin{displaymath}\begin{split}
        \{n \in \mathbb{N} | n < 0 \lor n^2 < 2\} \\
        \{n \in \mathbb{N} | n > 0 \land n^2 > 2\}
    \end{split}\end{displaymath}

    Beide Mengen sind klar definiert, enthalten aber weder gemeinsame Elemente noch $\sqrt{2}$
    (Da diese keine rationale Zahl ist).

    Wir wollen im folgendem die untere Menge als Dedekind'scher Schnitt bezeichnen.

    \section{Formale Defintion Dedekind'scher Schnitte}

    Ein Dedekind'scher Schnitte r ist eine Teilmenge der rationalen Zahlen $\mathbb{Z}$, welche nach oben Beschränkt ist, d.h.
    eine rationale Zahl q existiert, für die $\forall k \in r: k <= q$.
    Desweiteren muss gelten: $\forall a \in r: \forall b \in \mathbb{Z}: b <= a \rightarrow b \in r$

    \newpage

    \appendix

    \newpage

    \section{Literaturverzeichnis}

    Uwe Storch, Hartmut Wiebe: Grundkonzepte der Mathematik: Mengentheoretische, algebraische, topologische Grundlagen
    sowie reelle und komplexe Zahlen, Springer Spektrum Verlag

    Prof. Dr. Siegfried Echterhoff: Ergäanzung zur Vorlesung Analysis I WS08/0, Konstruktion von R,
    $https://ivv5hpp.uni-muenster.de/u/echters/Analysis/AnalysisI/Skript/Konstruktion_reelle_Zahlen.pdf$
    Zuletzt abgerufen am 02.10.2021

    \newpage

    \fbox{\parbox{\linewidth}{

    \textit{Ich erkläre hiermit, dass ich die Seminararbeit ohne fremde Hilfe angefertigt und nur die im
    Literaturverzeichnis angeführten Quellen und Hilfsmittel benützt habe.}

    \bigskip

    ........................................, den ......................................
    \\[-0.2cm]\hspace*{22mm} Ort \hspace*{55mm} Datum

    \bigskip
    \bigskip

    ................................................................................................................
    \\[-0.2cm]Unterschrift des Verfassers
    }}



\end{document}