%! Author = Sven
%! Date = 18.09.2021

% Preamble
\documentclass[12pt]{article}

% Packages
\usepackage{amsmath}
\usepackage{setspace}
\usepackage{parskip}
\usepackage{amsfonts}
\usepackage[ngerman]{babel}

\usepackage[a4paper, left=3.5cm, right=2.5cm, top=2.5cm, bottom=2cm]{geometry}

\onehalfspacing

% Document
\begin{document}

    \thispagestyle{empty}

    {
        \centering

        \begin{singlespace}

        \textbf{Hans-Sachs-Gymnasium Nürnberg}

        \textbf{Oberstufenjahrgang 2020/2022}

        \end{singlespace}

        \bigskip

        Seminarfach Mathematik

        \bigskip

        Seminararbeit

        \bigskip
        \bigskip

        \textbf{Unendliche und überabzählbare Mengen}

        \bigskip
        \bigskip

    }

    \hspace*{30mm}Verfasser: \hspace*{22mm} Sven Hoffart

    \bigskip

    \hspace*{30mm}Kursleiter: \hspace*{20mm} OStR Dr. Dennis Simon

    \bigskip

    \hspace*{30mm}Bewertung der Arbeit: \hspace*{10mm} \line(1, 0){35mm}

    \bigskip

    \hspace*{30mm}Bewertung der Präsentation: \line(1, 0){35mm}

    \bigskip
    \bigskip
    \bigskip
    \bigskip
    \bigskip

    \hspace*{30mm}Unterschrift Kursleiter: \hspace*{8mm} \line(1, 0){45mm}

    \bigskip

    \newpage

    \thispagestyle{empty}

    \tableofcontents

    \newpage

    \setcounter{page}{3}

    % Hier die Arbeit

    \section{Einleitung}

    \section{Satz von Cantor}

    Sei M eine Menge und $\mathcal{P}(M)$ ihre Potenzmenge.

    Der Satz von Cantor besagt nun, dass $|M| < |\mathcal{P}(M)|$

    \subsection{Beweis}

    % Hier endet die eigentliche Arbeit

    \newpage

    \appendix

    \newpage

    \begin{thebibliography}{xxxxxxxxxxxxxxxxxxx}
       \bibitem[BMBF, 2003]{bmbf}"'IT-Ausstattung der allgemein bildenden und berufsbildenden
                                   Schulen in Deutschland"', http://www.schulen-ans-netz.de/
                                   neuemedien/fakten/dokus/it-ausstattung-2003.pdf, 10.03.2005
    \end{thebibliography}

    \newpage

    \fbox{\parbox{\linewidth}{

    \textit{Ich erkläre hiermit, dass ich die Seminararbeit ohne fremde Hilfe angefertigt und nur die im
    Literaturverzeichnis angeführten Quellen und Hilfsmittel benützt habe.}

    \bigskip

    ........................................, den ......................................
    \\[-0.2cm]\hspace*{22mm} Ort \hspace*{55mm} Datum

    \bigskip
    \bigskip

    ................................................................................................................
    \\[-0.2cm]Unterschrift des Verfassers
    }}
\end{document}