%! Author = Sven
%! Date = 18.09.2021

% Preamble
\documentclass[12pt]{article}

% Packages
\usepackage{amsmath}
\usepackage{setspace}
\usepackage{parskip}
\usepackage{amsfonts}
\usepackage{xcolor}
\usepackage{amssymb}
\usepackage[ngerman]{babel}

\usepackage[a4paper, left=3.5cm, right=2.5cm, top=2.5cm, bottom=2cm]{geometry}

\onehalfspacing
% Document
\begin{document}

    \thispagestyle{empty}

    {
        \centering

        \begin{singlespace}

        \textbf{Hans-Sachs-Gymnasium Nürnberg}

        \textbf{Oberstufenjahrgang 2020/2022}

        \end{singlespace}

        \bigskip

        Seminarfach Mathematik

        \bigskip

        Seminararbeit

        \bigskip
        \bigskip

        \textbf{Unendliche und überabzählbare Mengen}

        \bigskip
        \bigskip

    }

    \hspace*{30mm}Verfasser: \hspace*{22mm} Sven Hoffart

    \bigskip

    \hspace*{30mm}Kursleiter: \hspace*{20mm} OStR Dr. Dennis Simon

    \bigskip

    \hspace*{30mm}Bewertung der Arbeit: \hspace*{10mm} \line(1, 0){35mm}

    \bigskip

    \hspace*{30mm}Bewertung der Präsentation: \line(1, 0){35mm}

    \bigskip
    \bigskip
    \bigskip
    \bigskip
    \bigskip

    \hspace*{30mm}Unterschrift Kursleiter: \hspace*{8mm} \line(1, 0){45mm}

    \bigskip

    \newpage

    \thispagestyle{empty}

    \tableofcontents

    \newpage

    \setcounter{page}{3}

    % Hier die Arbeit

    \section*{I. Mächtigkeit von Mengen}

		\subsection*{1. Arten von Abbildungen}
		
		\subsubsection*{a) Injektive Abbildungen}
		
		Eine Abbildung f: $A \rightarrow B$ heißt \underline{injektiv}, wenn es zu jedem Element b aus B 
		(Zielmenge) \underline{höchstens ein} Element a aus A (Definitionsmenge) gibt.
		
		Man schreibt dann: {\color{blue}f: $A \xrightarrow{\text{1-1}} B$} und sagt: {\color{blue}f ist 
		eine Injektion von A nach B}
		
		\underline{Beispiel im Mengendiagramm:}		
		
		% Mengediagramm!!!
		
		Für die Elemente $b_1$, $b_2$, $b_3$ gibt es \underline{genau} jeweils ein $a_1$, $a_2$, $a_3$
		aus A, für $b_4$ aus B gibt es aber \underline{kein} $a_4$ aus A ($a_4 \not \in A$).
		
		Zu jedem Element b aus B gibt es somit \underline{höchstens} ein Element a aus A. 
		Damit ist die Abbildung $f: A \rightarrow B$ in unserem Beispiel injektiv.
		
		\underline{Beispiele aus der Analysis:}
		
		f: $\mathbb{R} \rightarrow \mathbb{R}$ \hspace*{2mm} $x \mapsto x^2$ $\Rightarrow$ Diese Abbildung ist nicht injektiv
		
		g: $\mathbb{R}_0^+ \rightarrow \mathbb{R}$ \hspace*{2mm} $x \mapsto x^2$ $\Rightarrow$ Diese Abbildung ist injektiv
		
		h: $\mathbb{R} \rightarrow \mathbb{R}_0^-$ \hspace*{2mm} $x \mapsto x^2$ $\Rightarrow$ Diese Abbildung ist injektiv
		
		i: $\mathbb{C} \rightarrow \mathbb{R}_0^-$ \hspace*{2mm} $x \mapsto x^2$ $\Rightarrow$ Diese Abbildung ist injektiv
		
		\subsubsection*{b) Surjektive Abbildungen}
		
		Eine Abbildung f: $A \rightarrow B$ heißt \underline{surjektiv}, wenn es zu jedem Element b aus B
		\underline{mindestens ein} Element a aus A mit $f(a) = b$ gibt.
		
		Man schreibt dann: {\color{blue}f: $A \xrightarrow[\text{auf}]{} B$} und sagt: {\color{blue}f ist 
		eine Surjektion von A nach B}
		
		\underline{Formale Definition:} $(\forall b \in B) (\exists a \in A) (b = f(a))$
		
		Bei surjektiven Abbildungen gilt: $im_f = cod_f \Leftrightarrow B = im_f$.
		
		\underline{Beispiele im Mengendiagramm:}
		
		% Abbildung!!!!
		
		Diese Abbildung f: $A \leftarrow B$ ist surjektiv, da für jedes Element b aus B \underline{mindestens}
		ein Element a aus A existiert.

		\underline{Gegenbeispiel:}
		
		% Diagramm!!!!!!!
		
		Diese Abbildung ist weder \underline{injektiv} noch \underline{surjektiv}.
		
		\underline{Beispiele aus der Analysis:}
		
		f: $\mathbb{R} \rightarrow \mathbb{R}$ \hspace*{2mm} $x \mapsto x^2$ $\Rightarrow$ Die Abbildung ist nicht surjektiv
		
		g: $\mathbb{R} \rightarrow \mathbb{R}_0^+$ \hspace*{2mm} $x \mapsto x^2$ $\Rightarrow$ Die Abbildung ist surjektiv
		
		\subsubsection*{c) Bijektive Abbildungen}
		
		Existiert zu jedem b aus B \underline{genau ein} Element a aus A, so ist die Abbildung f: $A \rightarrow B$ sowohl
		\underline{injektiv} als auch \underline{surjektiv}. Man sagt dann: {\color{blue}f ist \underline{bijektiv}}
		
		Man schreibt dann: {\color{blue}f: $A \xrightarrow[\text{auf}]{1-1} B$} und spricht: {\color{blue}f ist eine Bijektion von A nach B}
		
		Bijektivität ist der Schnitt der Eigenschaften aus Injektivität und Surjektivität.
		
		\underline{Beispiel im Mengendiagramm:}
		
		% Diagramm!!!!!
		
		\underline{Beispiel aus der Analysis:} \hspace*{3mm} f: $\mathbb{R} \rightarrow \mathbb{R}$ \hspace*{2mm} $x \mapsto x^3$
		
		\underline{Fazit:} Ist eine Abbildung f: $A \rightarrow B$ bijektiv, so enthalten die Mengen A und B jeweils dieselbe Anzahl 
		an Elementen.
		
		\subsection*{2. Mächtigkeit von Mengen}
		\subsubsection*{a) Gleichmächtige Mengen}
		
		\fbox{\parbox{\linewidth}{
		\underline{Definition:}
		
		Zwei Mengen A und B heißen genau dann \underline{gleichmächtig}, wenn es eine Bijektion von A nach B gibt.
		
		\bigskip
		
		Notation: $A \approx B$
		
		Formal: $A \approx B \hspace*{1mm} \Leftrightarrow \hspace*{1mm} \exists (f) (f: A \xrightarrow[\text{auf}]{1-1} B)$
		}}
		
		\bigskip
		
		Wenn A und B gleichmächtig sind, enthalten beide Mengen \underline{gleich viele} Elemente.
		Haben zwei Mengen diesselbe Anzahl an Elementen, so lässt sich in jedem Fall eine Bijektion zwischen ihnen herstellen,
		sodass diese gleichmächtig zueinander sind.
		
		\underline{Wichtige Sätze:}
		
		{
			\begin{singlespace}
			a) $A \approx A$
			
			b) $A \approx B \Rightarrow B \approx A$ \hspace*{3mm} (Kommutativität)
			
			c) $(A \approx B \land B \approx C) \Rightarrow A \approx C$
			\end{singlespace}
		}
		
		\bigskip
		
		\underline{Beweise:}
		
		a) Die Identitätsfunktion, die jedes Element a aus A auf die Elemente $I_A$ aus A 
		abbildet, ist definiert als $I_A(x) = x$.
		Somit liefert $I_A$ eine Bijektion von A nach A (zu sich selbst). \hspace*{2mm} q.e.d.
		
		b) Wenn f: $A \xrightarrow[\text{auf}]{1-1} B$ gilt, dann folgt daraus $f^{-1}$: $B 
		\xrightarrow[\text{auf}]{1-1} A$ \hspace*{1mm}$\Leftrightarrow \hspace*{1mm} B \approx A$. \hspace*{2mm} q.e.d.
		\footnote{
			Die inverse Funktion zu einer bijektiven Funktion ist ebenfalls bijektiv.
			Vgl. F. Kahlhoff et all, Skript zu den Vorlesungen Mathematik für Informatiker I und II, 2009/10, 
			page.mi.fu-berlin.de/baumeist/Mafi-Skript.pdf, zuletzt abgerufen am 3.11.2021, 19:05, S. 16
		}
		
		c) Wenn f: $A \xrightarrow[\text{auf}]{1-1} B$ und g: $B \xrightarrow[\text{auf}]{1-1} C$ gilt, dann
		folgt daraus: $g \circ f$: $A \xrightarrow[\text{auf}]{1-1} C$ \hspace*{1mm}$\Leftrightarrow \hspace*{1mm} A \approx C$. \hspace*{2mm} q.e.d.
		\footnote{
			Die Komposition zweier bijektiven / injektiven Funktionen ist ebenfalls bijektiv / injektiv. Vgl. S. 15
		}
		
		\subsubsection*{b) Ungleichmächtige Mengen}
		
		\fbox{\parbox{\linewidth}{
			\underline{Definition:} {\color{blue}Höchstens gleichmächtige Mengen}
			
			Genau dann, wenn es eine Injektion von A nach B gibt,
			heißt A \underline{höchstens gleichmächtig} zu B.
			
			\bigskip
			
			Notation: $A \precsim B$ (,,A höchstens gleichmächtig zu B'')
			
			Formal: $A \precsim B \hspace*{1mm} \Leftrightarrow \hspace*{1mm} \exists (f) (f: A \xrightarrow{1-1} B)$
		}}
		
		Ist A höchstens gleichmächtig zu B, so hat die Menge B \underline{mindestens}
		so viele Elemente wie die Menge A bzw. die Menge A \underline{höchstens} so viele
		Elemente wie B.
		
		\bigskip
		
		\fbox{\parbox{\linewidth}{
			\underline{Definition:} {\color{blue}Weniger mächtige Mengen}
			
			Die Menge A heißt genau dann \underline{weniger Mächtig} als B, wenn
			es eine Injektion von A nach B gibt, aber zugleich keine Bijektion 
			von A nach B existiert.
			
			\bigskip
			
			Notation: $A \prec B$ (,,A weniger mächtig als B'')
			
			Formal: $A \prec B \hspace*{1mm} \Leftrightarrow \hspace*{1mm} (A \precsim B) \land (A \not \approx B)$
		}}
		
		Ist A weniger mächtig als B, so hat die Menge B mehr Elemente als A bzw. die Menge A weniger als B.
		
		\underline{Wichtige Sätze:}
		
		{
			\begin{singlespace}
				a) $A \precsim A$
				
				b) $A \precsim B \land B \precsim C \Rightarrow A \precsim C$
				
				c) A \subseteq B \Rightarrow A \precsim B
				
				d) A \precsim B \Leftrightarrow (A \prec B \lor A \approx B)
			\end{singlespace}
		}
		
		\underline{Beweise:}
		
		a) Die Identitätsfunktion liefert eine Bijektion von A nach A, 
		wodruch hier gleichzeitig die Injektion und Surjektion von A nach A 
		gewährleistet wurde. Also gilt: $A \precsim A$. \hspace*{2mm} q.e.d.
		
		b) Wenn f: $A \xrightarrow{1-1} B$ und g: $B \xrightarrow{1-1} C$ gilt, dann folgt daraus
		$g \circ f$: $A \xrightarrow{1-1} C$.$^2$ $\blacksquare$
		
		c) Wenn $A \subseteq B$ ist, dann bildet die Identitätsfunktion $I_A$ jedes Element a aus A 
		jeweils auf dieselben Elemente a aus B ab. Für den Fall, dass $A \subset B$ gilt: 
		Für alle anderen Elemente b aus B, die nicht in a sind, gibt es kein Element aus A.
		$\forall a, b \in B$ gibt es höchstens ein Element a aus A.
		Daraus folgt: $I_A$: $A \xrightarrow{1-1} B \Leftrightarrow A \precsim B$.
		
		d) Wenn $A \precsim B$ gilt, so ist $A \approx B$ per Definition genau dann, wenn $A \not \prec B$ ist.
		Im Gegensatz dazu ist $A \precsim B$ offensichtlich, wenn $A \prec B$ oder $A \approx B$ ist.
		
		\subsection*{3. Der Satz von Cantor}
		
		\fbox{\parbox{\linewidth}{
			Jede Menge A ist weniger mächtig als ihre eigene Potenzmenge $\mathcal{P}(A)$:
			\[
				A \prec \mathcal{P}(A)
			\]
		}}
		
		\subsubsection*{Beweis}
		
		$A \prec \mathcal{P}(A)$ gilt per Definition genau dann, wenn 
		$A \precsim \mathcal{P}(A) \land A \not \approx \mathcal{P}(A)$.
		
		Im Folgenden zeigen wir nun, dass $A \precsim \mathcal{P}(A)$ und 
		$A \not \approx \mathcal{P}(A)$ gelten.
		
		\underline{1) $A \precsim \mathcal{P}(A)$:}
		
		$A \precsim \mathcal{P}(A) \hspace*{1mm} \Leftrightarrow \hspace*{1mm} \exists (f) (f: A \xrightarrow{1-1} \mathcal{P}(A))$
		
		Wir suchen eine Injektion f von A nach $\mathcal{P}(A)$.
		Für jedes Element x in A lässt sich jeweils eine Teilmenge $\{x\} \subseteq A$ bilden.
		Jeweils jede dieser Teilmengen $\{x\}$ ist zugleich jeweils eine Element aus $\mathcal{P}(A)$.
		Also $\{x\} \in \mathcal{P}(A)$.
		$\mathcal{P}(A)$ hat mindestens genau soviele Element wie A selbst. 
		Jedes Element x aus A wird durch die Injektion f: $A \rightarrow \mathcal{P}(A)$ mit
		$x \mapsto \{x\}$ jeweils auf ein anderes Element $\{x\}$ aus $\mathcal{P}(A)$ abgebildet.
		
		$\blacksquare$
		
		\underline{2) $A \not \approx \mathcal{P}(A)$:}
		
		Wir haben vorher gezeigt, dass $A \precsim \mathcal{P}(A)$ bzw. dass die Abbildung 
		f: $A \rightarrow \mathcal{P}(A)$ injektiv ist.
		Diese Erkenntnis schließt jedoch nicht die möglichkeit aus, dass die Abbildung 
		gleichzeitig auch bijektiv sein könnte.
		Wir zeigen also, dass es keine Bijektion von A nach $\mathcal{P}(A)$ gibt,
		d.h. $A \not \approx \mathcal{P}(A)$.
		
		Dazu nehmen wir an, dass $A \approx \mathcal{P}(A)$ gelte, und es deshalb eine
		bijektive Abbildung g: $A \xrightarrow{1-1} \mathcal{P}(A)$ gäbe.
		
		Wir definieren eine Menge $B = \{y \in A: y \not \in g(y)\}; \hspace*{1mm} g(y) \in \mathcal{P}(A)$
		
		% Abbildung!!!
		
		B enthält alle Elemente y aus A, die in ihrem Funktionswert g(y) liegen.
		B muss also eine Teilmenge von A sein: $A \subseteq B$.
		B ist dann auch ein Element aus $\mathcal{P}(A)$: $B \in \mathcal{P}(A)$
		
		Es gilt: $B = g(y)$ für \underline{einige} y aus A.
		
		Per Definition der Menge B sind alle Element y aus B keine Elemente von $g(y)$.
		Also $y \in B \Leftrightarrow < \not \in g(y)$.
		
		Da $B = g(y)$ für einige y aus A gilt, gibt es y aus B die gleichzeitig in $g(y)$ sind.
		$\exists y \in B: y \in g(y)$
		
		Per Definition gibt es keine y in B, die gleichzeitig in $g(y)$ enthalten sind.
		
		Wir erhalten einen Widerspruch.
		Also muss unsere Annahme $A \approx \mathcal{P}(A)$ falsch sein.
		$\blacksquare$
		
		Damit haben wir die Gültigkeit des Satzes von Cantor bewiesen.
		$\blacksquare$
		
		\section*{II. Endliche Mengen}
		
		\subsection*{1. Endliche Mengen}
		
		Wir definieren folgende Menge: $P_n = \{k \in \mathbb{N} | k \le n\}$ mit $n \in \mathbb{N}$
		
		In der Menge $P_n$ sind alle Elemente k aus $\mathbb{N}$ enthalten, die kleiner gleich n aus
		$\mathbb{N}$ sind. $P_n$ lässt sich in der Form als:
		\[
			P_n = \{1;2;3;...;n-1;n\}
		\]
		darstellen.
		
		n gibt dabei die Anzahl der Elemente der Menge P an. $P_n$ hat also n Elemente und somit 
		\underline{endlich} viele Elemente.
		
		\underline{Beispiele zur Vorstellung:}
		
		$P_1 = \{1\}$ hat 1 Element
		
		$P_2 = \{1; 2\}$ hat 2 Elemente
		
		$P_3 = \{1; 2; 3\}$ hat 3 Elemente
		
		\bigskip
		
		\fbox{\parbox{\linewidth}{
			\underline{Definition:}
			
			Eine beliebige Menge A heißt genau dann \underline{endlich}, wenn 
			sie leer ist oder wenn es eine natürliche Zahl n aus $\mathbb{N}$ gibt,
			sodass die Abbildung f: $A \rightarrow P_n$ bijektiv ist bzw. $A \approx P_n$ gilt.
			
			\bigskip
			Formal: A ist endlich $\Leftrightarrow A = \emptyset \lor \exists n \in \mathbb{N}: (A \approx P_n)$
			
			\bigskip
			\underline{\textbf{Fazit}:} Jede Menge A besitzt jeweils genau so viele Elemente wie jeweils ihre zugehörige
			gleichmächtige Menge $P_n$ und somit \underline{endlich viele} Elemente.
			
			Einfach gesagt folgt daraus, dass eine Menge A genau dann endlich heißt, wenn sie endlich
			viele Elemente hat.
		}}
		
		\bigskip
		
		\subsection*{2. Unendliche Mengen}
		
		\fbox{\parbox{\linewidth}{
			\underline{Definition:}
					
			Eine beliebige Menge A heißt genau dann unendlich, wenn sie nicht endlich ist.
			
			\bigskip
			
			Formal: A ist unendlich $\Leftrightarrow \lnot (A = \emptyset \lor \exists (n \in \mathbb{N}): (A \approx P_n)$
			$\Leftrightarrow a \not = \emptyset \land \forall (n \in \mathbb{N}): (A \not \approx P_n)$
			
			\bigskip
			\underline{\textbf{Fazit}:} Jede unendliche Menge A ist zu keiner einzigen endlichen Menge $P_n$ mit endlich vielen Elementen n gleichmächtig.
			
			Es folgt daraus: Jede unendliche Menge A besitzt unendlich viele Elemente, d.h. jede Menge A heißt genau dann
			unendlich, wenn sie unendlich viele Elemente hat.
		}}
		
		\subsection*{3. Eigenschaften endlicher und unendliche Mengen}
		
		\subsubsection*{a) Endliche Mengen}
		
		{
			\begin{singlespace}
				1) Jede endliche Menge, die gleichmächtig zu einer endlichen Menge ist, ist auch endlich.
				
				2) Wenn A endlich ist, so ist $A \cup \{y\}$ auch endlich
				
				3) Jede Teilmenge D einer endlichen Menge B ist endlich. 
				
				4) Wenn B endlich ist und $D \precsim B$ gilt, dann ist D ebenfalls endlich.
				
				5) Wenn A eine endliche Menge und B eine beliebige Menge ist, dann sind die Mengen $A \cap B$ und $A \backslash B$ auch endlich.
				
				6) Die Vereinigung endlich vieler Mengen ist wieder eine endliche Menge
				
				7) Jede endliche Menge B ist nicht gleichmächtig zu ihren echten Teilmengen.
				
				8) Wenn A und B endlich sind, so sind auch $A \cup B$ endlich.
			\end{singlespace}
		}
		
		\underline{Beweise:}
		
		1) ohne Beweis
		
		2) a) $A = \emptyset$: $A \cup \{y\} = \{y\} \approx P_1$ ist endlich
		
		b) $A \approx P_n$: 
		
		$\alpha$ $y \in A$: $A \cup {y} = A \approx P_n$ ist endlich
		
		$\beta$ $y \not \in A$: $A \cup {y} \approx P_{n+1}$ ist endlich
		
		3) Vgl. 'Number Systems and the Foundation of Analysis',  S. 299
		
		4) Es gilt: B ist endlich und $D \precsim B$. Dann gibt es eine Injektion von D nach B und das bedeutet, dass es eine Bijektion von D auf eine 
		Teilmenge von B gibt. Diese Teilmenge von B ist nach Satz 3 endlich und daraus folgt, dass D nach Satz 1 endlich ist.
		
		5) $A \cap B$ und $A \backslash B$ sind Teilmengen der endlichen Menge A und somit endlich.
		
		6) Verzicht auf Beweis. Beweis wird durch das Prinzip von Inklusion und Exklusion ausgeführt.
		
		7) 
		
		\underline{Annahmen}
		
		1) $B = \emptyset$: B hat keine echten Teilmengen
		
		2) $B \approx P_n$: Die Funktion $\varphi$ liefiert eine Bijektion von B nach $P_n$, d.h. $\varphi$: $B \xrightarrow[auf]{1-1} P_n}$
		
		Sei C eine beliebige Teilmenge von B.
		Wir müssen jetzt zeigen, dass $C \not \approx B$ gilt:
		
		\[\varphi\text{: } C \xrightarrow[auf]{1-1} \varphi[C] \text{ mit } \varphi[C] = \{\varphi(c) | c \in C\}\]
		
		Die Menge $\varphi[C] = \{\varphi(c) | c \in C\} \subset P_n$ ist nicht gleichmächtig
		zu $P_n$, d.h. $\varphi[C] \not \approx P_n$.\footnote{
			Diese Schlussfolgerung ist nach folgendem Satz zulässig:
			
			$\forall A | (A \subset P_n): A \not \approx P_n$
			
			Beweis: 'Number Systems and the Foundation of Analysis', S.298, Lemma D6
		}
		
		Da $\varphi[C] \approx C$ gilt, folgt daraus $C \not \approx P_n$ bzw. $C \not \approx B$.
		
		8) Vgl. 'Number Systems and the Foundation of Analysis', S. 301
		
		\subsubsection*{b) Unendliche Mengen}
		
		1) Jede Menge, die gleichmächtig zu einer unendlichen Menge ist, ist auch unendlich
		
		2) Wenn die Menge D unendlich ist und D eine Teilmenge von B, dann ist B ebenfalls unendlich.
		
		3) Wenn D unendlich ist, und $D \precsim B$ gilt, dann ist B auch unendlich.
		
		4) Die Menge P der natürlichen Zahlen ist gleichmächtig zu einer richtigen Teilmenge von P selbst.
		
		5) Unendliche Mengen besitzen sowohl unendliche als auch endliche Teilmengen.
		
		\underline{Beweise:}
		
		1) Ohne Beweis
		
		2) % ....
		
		3) Wenn D unendlich ist, und es eine Injektion von D nach B gibt, so gibt es immer eine Bijektion 
		von D auf eine Teilmenge von B.
		Diese Teilmenge von B ist nach Satz 1 unendlich und somit ist B nach Satz 2 auch unendlich.
		
		4) $P \backslash \{1\}$ ist eine Teilmenge von P, die gleichmächtig zu P selbst ist:
		
		Die Bijektion $\varphi$: $P \xrightarrow[auf]{1-1} P \backslash \{1\}$ ist durch den Funktionsterm
		$\varphi(n) = n + 1$ gegeben.
		
		5) Ohne Beweis
		
		\section*{III. Abzählbare und Überabzählbare Mengen}
		
		\subsection*{1. Abzählbare Menge}
		
		Eine beliebige Menge A heißt genau dann...
		
		a) ... endlich abzählbar, wenn sie endlich ist und somit endlich viele Elemente hat. Ihre Elemente lassen
		   sich mithilfe  der Natürlichen Zahlen ,,abzählen'' bzw. ,,durchnummerieren''.
		
		b) ...unendlich abzählbar, wenn sie unendlich ist und es dabei eine Bijektion von A auf die Menge $\mathbb{N}$ der natürlichen Zahlen
		gibt bzw. A gleichmächtig zu $\mathbb{N}$ ist.
		
		\underline{Also:}
		A ist unendlich abzählbar $\Leftrightarrow (\exists f): f: A \xrightarrow[auf]{1-1} \mathbb{N} \Leftrightarrow A \approx \mathbb{N}$
		
		Jede unendliche abzählbare Menge A ist gleichmächtig zur Menge der natürlichen Zahlen. Die Menge A hat in dem Sinne dann genauso viele unendliche
		Elemente wie $\mathbb{N}$. Ihre Elemente lassen sich dann aufgrund ihrer bijektiven Beziehung zu $\mathbb{N}$ mithilfe der natürlichen Zahlen ,,ins Unendliche''
		durchnummerieren bzw. die natürlichen Zahlen reichen dann genau aus, alle Elemte von A abzuzählen.

    % Hier endet die eigentliche Arbeit

    \newpage

    \appendix

    \newpage
		
		- F. Kahlhoff et all, Skript zu den Vorlesungen Mathematik für Informatiker I und II, 2009/10, 
			page.mi.fu-berlin.de/baumeist/Mafi-Skript.pdf, zuletzt abgerufen am 3.11.2021, 19:05

    \begin{thebibliography}{xxxxxxxxxxxxxxxxxxx}
       \bibitem[BMBF, 2003]{bmbf}"'IT-Ausstattung der allgemein bildenden und berufsbildenden
                                   Schulen in Deutschland"', http://www.schulen-ans-netz.de/
                                   neuemedien/fakten/dokus/it-ausstattung-2003.pdf, 10.03.2005
    \end{thebibliography}

    \newpage

    \fbox{\parbox{\linewidth}{

    \textit{Ich erkläre hiermit, dass ich die Seminararbeit ohne fremde Hilfe angefertigt und nur die im
    Literaturverzeichnis angeführten Quellen und Hilfsmittel benützt habe.}

    \bigskip

    ........................................, den ......................................
    \\[-0.2cm]\hspace*{22mm} Ort \hspace*{55mm} Datum

    \bigskip
    \bigskip

    ................................................................................................................
    \\[-0.2cm]Unterschrift des Verfassers
    }}
\end{document}